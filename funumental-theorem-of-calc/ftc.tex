\documentclass{article}

\usepackage{amsmath}

\title{The Fundumental Theorem of Calculus}
\author{Aiden Wenzel}

\begin{document}
\maketitle 

The Fundumental Theorem of Calculus is a theorem which links the concepts of differentiation and integration of functions. 
The Fundumental Theorem of Calculus is broken into two parts. 

\section{The First Fundamental Theorem of Calculus}
\subsection{Definition}
If $f(x)$ is continuous over an interval $[a,b]$, and the function $F(x)$ is defined by 
\[F(x) = \int_a^x f(t)dt,\]
then $F'(x) = f(x)$ over $[a,b]$. \textbf{NEED CITATION FROM LIBRE TEXT}

\subsection{Proof}
We must prove that \(F'(x) = f(x)\). 

By the definition of a derivative, and the definition of \(F(x)\) given in the theorem,

\begin{equation}
\begin{split}
    F'(x) &= \lim_{h \to 0}{\frac{F(x+h)-F(x)}{h}} \\
          &= \lim_{h \to 0}{\frac{1}{h} (\int_a^{x+h} f(t)dt - \int_a^x f(t)dt)} \\
    F'(x) &= \lim_{h \to 0}{\frac{1}{h} \int_x^{x+h} f(t)dt}
\end{split}
\end{equation}

\(\frac{1}{h} \int_x^{x+h} f(t)dt\) is in the form of the average value of a function where,

\[f_{avg} = \frac{1}{b-a} \int_b^a f(x)dx\]




\end{document}
